\chapter{Conclusions and Future Work}
\label{conclusions_future_work}

This final chapter includes a brief summary of all the topics covered on this thesis, highlighting the most important aspects of it, as well as some suggestions of topics to be developed as future work.\\

\section{Conclusions}
\label{conclusions}



We began the thesis presenting the main topic, modular robots, and their main terminology and applications. We also explained what are the most frequent problems found when working with modular robots: their high number of degrees of freedom, that make very difficult the design of efficient gaits with analytical methods; their distributed nature, that require a distributed controller, in which each module only has information about its neighbours and control of the whole modular robot emerges from the sum of all the individual decisions; and the high cost of modular robots, due to the large number of modules required and the  high cost of each of them.\\

We presented our main objectives, that correspond with solutions to those main problems. The first one is to obtain optimal locomotion gaits with a bio-inspired approach, using sinusoidal oscillators and optimizing their main parameters (amplitude, offset, phase and frequency) with a evolutionary optimization algorithm called Differential Evolution, which is based on biological evolution. The second one is to design a homogeneous distributed controller that is able to discover the current global configuration of the modular robot and the role of the module inside that configuration using digital hormones. The last one is to design a cheap modular robotic platform to test the gaits and controller obtained and validate them in a real robot, apart from the simulated one.\\

Once these objectives were presented, we reviewed the state of the art in the topics covered by the main problems. We introduced the main existing modules and their features, including the module Y1 designed by Juan Gonzalez-Gomez, in which we based the design of our module and electronic control board. We also reviewed the different techniques used for locomotion in modular robots, such as gait tables, CPGs and sinusoidal oscillators. Finally, we introduced the problem of coordination in modular robotics, and the existing solutions in the literature.\\

All the software framework implemented for modular robots, named \emph{`Hormodular'}, and the methodology used to develop it, called `Test-Driven Development' were presented in the next chapter. This methodology consists in developing code with the help of unit tests, that test the requirements of the software to be implemented, and help the programmer to develop code faster, in a more maintainable way, and writing the minimum amount of code necessary to accomplish the requirements of the project. We also presented the different software dependencies of the project, a detailed explanation of each class belonging to the `Hormodular' framework and the instructions to download and run the software, which has been released publicly under a open source license.\\

Afterwards, we presented the hardware platform developed to test the locomotion gaits and the distributed controller. We explained the main features of the Y1 module and their derivatives, as well as the Skymega board, in which we based our work. We introduced the drawbacks found on these platforms and explain how we solved them on our designs. Our module, the REPY-2, can be produced at a low cost and is easy to manufacture using a 3D printer. Improves the previous design with a symmetrical, more resistent design, that allows not only linear configurations, but also 2D configurations. Detailed instructions to assemble this module were also included. The electronic board designed, the SkymegaSMD, solves some flaws of the previous Skymega design by using SMD components, which allowed us to include a 5V regulator, split the power supply of the servos and improve the circuit layout, reducing the electrical noise. We also talked about the other components required to assemble the modular robot, such as the batteries or the USB/serial adapter, and about the firmware implemented to control the robot from the computer.\\

The next topic discussed was the modular robot configuration and gaits, in which we introduced the different types of configurations allowed by the REPY-2 modules design, as well as our method for describing these different configurations. This 
method encodes the connections of one module in a ID that represents them, using for that purpose the IDs of the connectors involved in the connection, as well as the relative rotation between them. We also explain the sinusoidal oscillators, selected to achieve the locomotion and coordination of the modular robot by using a simplified version of the CPG model, which is more complex and demands more computing power. Finally, we introduced the Differential Evolution algorithm that we used to find the parameters of the sinusoidal oscillators which yield optimal locomotion gaits for the three configurations considered in this work.\\

Next, we introduce the concept of digital hormone, comparing them to the biological hormones, as a message that does not have a fixed destination but floats on the distributed system, has a lifetime, and triggers different actions depending on its receiver. We explain the structure of our digital hormones, as well as the hormone communication protocol. This protocol, based in three types of hormone (`ping', `leg' and `head') is used by the modules to discover the global configuration of the modular robot, as well as the role of each module inside that configuration. With this information, each module can select the appropiate parameters for their sinusoidal oscillators, from the ones learned with the evolutionary algorithms. \\

To conclude, we present the results obtained when testing the locomotion gaits and the homone-based controller. The resulting gaits are then analyzed, as well as their trajectory and speed, and their relationship to the sinusoidal oscillator parameters used to produce them. We observe how these parameters have generated stable and fast gaits, that can be reproduced in the real modular robot with a similar performance for surfaces with a friction coefficient high enough. Finally, we tested the hormone controller for the three configurations and we checked how after a short period of time the modules are able to discover their global configuration and their role inside that configuration, and they are able to select the appropiate parameters that have been previously found with the differential evolution optimization.\\

This way we checked that all the objectives proposed for the thesis have been accomplished: we found optimal gaits for the three configurations using sinusoidal oscillators whose parameters were optimized through differential evolution; we designed a homogeneous distributed controller based on digital hormones that is able to discover the modular robot global configuration as well as the function of the module inside that configuration, and select accordingly the parameters for the sinusoidal oscillators; and we developed a cheap modular robotic platform that was used to test our work, and that can be reused for other modular robotics researches. \\

%%%%%%%%%%%%%%%% FUTURE WORK %%%%%%%%%%%%%%%%%%%%%%%%%%%%%%%%%%%%%%%%%%%%%%%%%%%%%%%%%%%%%%%%%%%%%%%%%%%%%%%%
\newpage
\section{Future work}
\label{future_work}

In this section we will present some suggestions of possible improvements to the current work that could not be developed for this thesis due to limitations in time or resources, and that could be developed as future work.\\

\begin{enumerate}
	\item \textbf{Enable \robotNine and \robotEleven configurations on hardware plaftform.}\\ Due to hardware limitations, the current platform only allows using up to 8 servos with a single SkymegaSMD board, which only allows us to test the gaits on the \robotSeven configuration. For the remaining configurations, that count with 9 and 11 servos respectively, two SkymegaSMD boards have to be used.
	
	These boards have to be connected using the I2C bus, and the current firmware has to be extended to allow this communication through the I2C bus, and to select from the computer controller which board is to receive the joint values to be set to the servos.\\
	
	\item \textbf{Development of more advanced modules.}\\ The current platform is cheap and useful for testing different gaits and controllers on a real modular robot, but it is also very limited for other topics related to modular robotics. If a distributed controller or communications between modules have to be implemented, they mush be emulated on the computer, since the current robot has a central controller that can only receive the joint position values and set them on the different servos. Reconfiguration of the modular robot can only be achieved manually and, since the connectors use screws, this manual reconfiguration is very slow and tedious.
	
	 One possible improvement to the current work would be to develop a better modular robotic platform. This platform would need to have the control electronics, communications and power on each of the modules, so that the algoritms developed for them can be tested on the actual modules without the need for a computer. It would also have to feature a new connector that allows self-reconfiguration or, at least, that eases the manual reconfiguration process providing a simple lock/unlock mechanism.\\
	
	\item \textbf{Improvement of the current hormone-based communication protocol.} \\ The current hormone protocol works correctly with the three proposed configurations, but it relies too much on the particular aspects of each of them and, if a new configuration is to be added to the controller, it is not trivial to modify it to add the new configuration and its corresponding parameters to the controller.  
	
	The hormone protocol could be improved to a more generic one, in which locomotion gaits for new configurations can be added by adding the corresponding new parameter tables to the controller, and the hormone protocol can discover this new configuration and use its parameters without further modifications.\\
	
	\item \textbf{Add support for reconfiguration to the Hormodular framework.}\\ Hormodular currently does not support changing from one configuration to other one while the simulation is running, that being the reason why we only tested the hormone controller on each of the configurations individually.
	
	 In order to ensure that the homone controller also works when the robot configuration has changed, and to allow research related to self-reconfiguration on modular robots, this support to reconfiguration should be added to the software framework and simulator.\\
	
	\item \textbf{Add support for concurrent execution of controllers.}\\ To simplify the development of the controllers, the current version of Hormodular executes them sequentially, emulating the concurrency that it would exist if run on the different modules. This simplifies the process of developing and testing a controller, but the resulting controller is not realistic, and lacks some of the problems of actual distributed controllers, such as the need for synchronization and a robust communication protocol.
	
	If the controller of each module is run concurrently in simulation, all those aspects can be evaluated in conditions closer to the ones existing on the real life modular robot. \\
	
	\item \textbf{Add sensors to the modules or to the modular robot.}\\ In the current version of the controller, the sinusoidal oscillators are running using the optimized parameters in open loop, with any kind of sensorial feedback. Different kinds of sensors could be added to the modules, such as potentiometers or encoders to measure the actual joint position, inertial sensors (IMUs) to measure the movement of the module or IR / ultrasonic rangefinder to measure the distance to the possible obstacles the robot may face.
	
	 Using the information received from the sensors, and integrating it to the hormone communication stream, it would be possible to develop a reactive controller to modify the modular robot gaits to avoid obstacles or adapt the gait to changes in its performance due to changes in the terrain conditions (changes in friction coefficient, slope, etc).\\
\end{enumerate}