Modular robots are robots composed of multiple units, called `modules'. Each module is an independent robot, with its own control electronics, actuators, sensors, communications and power. These modules can change their position and configuration in order to adapt to the requirements of the situation, making modular robot suitable for tasks that involve unknown or unstructured terrains, in which a robot cannot be designed specifically for them. Some examples of those applications are space exploration, battlefield reconnaissance, finding victims among the debris in natural catastrophes and other similar tasks involving complicated terrains, which require a high versability.\\

But this versability comes with several drawbacks. As modular robots are composed of several independent robots, the nature of their controller is distributed, which difficults their design and programming, requiring additionally a robust communication protocol to share information among modules. The high number of modules also results in a robot with a with number of degrees of freedom, for which achieving the coordination required for locomotion becomes increasingly difficult. Finally, as the modules are fully independent robots, the cost of researching modular robotics is usually very high, since the price of building a single robot has to be multiplied by the high number of modules.\\

This thesis addresses those three mentioned problems: obtaining optimal locomotion gaits from a biologically inspired approach, using sinusoidal oscillators whose parameters are found through evolutionary optimization algorithms; developing a homogenous, distributed controller based on digital hormones that can recognize the current robot configuration and select the proper gait; and the development of a low-cost modular robotic platform to reseach locomotion gaits for different configurations.